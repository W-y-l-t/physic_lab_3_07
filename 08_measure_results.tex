\begin{itemize}
    \item Прилагаются \hyperlink{table2}{Таблица 2}, \hyperlink{table2}{Таблица 3}, \hyperlink{table2}{Таблица 4}  в Приложении 2

    \smallvspace

    \item Расчёт масштабирующих коэффициентов $\alpha$ и $\beta$: \\
    $\alpha = \frac{N_1}{\ell R_1} = \frac{1665}{0,078 \cdot 68} = 313,914 (\text{м} \cdot \text{Ом})^{-1}$ \\
    $\beta = \frac{R_2 C_1}{N_2 S} = \frac{47000 \cdot 4,7 \cdot 10^{-7}}{970 \cdot 6,4 \cdot 10^{-5}} = 3,558 \frac{\text{Ом} \cdot \text{Ф}}{\text{м}^2}$
    
    \smallvspace

    \item Расчёт коэрцитивной силы $H_c$ и остаточной индукции $B_r$: \\
    $H_c = \alpha \cdot K_x \cdot X_c = 313,914 \cdot 0,1 \cdot 1 = 31,391 \frac{\text{А}}{\text{м}}$ \\
    $B_r = \beta \cdot K_y \cdot Y_r = 3,558 \cdot 0,05 \cdot 1,4 = 0,249 \text{Тл}$

    \smallvspace

    \item Расчёт напряжённости $H_m$ и индукции $B_m$ магнитного поля, магнитной проницаемости $\mu$: \\
    $H_m = \alpha \cdot K_x \cdot X_m = 313,914 \cdot 0,1 \cdot 2,7 = 84,757 \frac{\text{А}}{\text{м}}$ \\
    $B_m = \beta \cdot K_y \cdot Y_m = 3,558 \cdot 0,05 \cdot 2,3 = 0,409 \text{Тл}$ \\
    $\mu = \frac{B_m}{\mu_0 H_m} = \frac{0,409}{4 \cdot \pi \cdot 10^{-7} \cdot 84,757} = 3842$

    \smallvspace

    \item Расчёт коэффициента $\chi$: \\
    $\chi = K_x K_y\frac{N_1 R_2 C_1}{N_2 R_1}f = 0,1 \cdot 0,05 \cdot \frac{1665 \cdot 470000 \cdot 4,7 \cdot 10^{-7}}{970 \cdot 68} \cdot 35 = 9,8 \cdot 10^{-4} \frac{\text{Дж}}{\text{с}}$

    \smallvspace

    \item Расчёт площади петли гистерезиса ферромагнетика $S_{\text{ПГ}}$: \\
    Сделаем интерполяцию верхней и нижней кривой петли гистерезиса при помощи полиномов Лагранжа $8$ степени: \\
    $y_{\text{верхн.}}(x) = -\frac{112694375}{70563397632}x^{8}+\frac{1581874625}{211690192896}x^{7}+\frac{64101575}{2520121344}x^{6}-\frac{234701845}{1890091008}x^{5}-\frac{3018111}{52502528}x^{4}+\frac{160856177}{295326720}x^{3}-\frac{126896811}{287123200}x^{2}+\frac{242428759}{538356000}x+\frac{3}{2}$ \\
    $y_{\text{нижн.}}(x) = \frac{889124375}{211690192896}x^{8}+\frac{264950125}{23521132544}x^{7}-\frac{141572225}{2520121344}x^{6}-\frac{31610635}{210010112}x^{5}+\frac{9677683}{52502528}x^{4}+\frac{4758729}{8203520}x^{3}+\frac{279651061}{1292054400}x^{2}+\frac{9812577}{22431500}x-\frac{7}{5}$ \\

    $S_{\text{ПГ}} = \int_{-3,2}^{2,7}(y_{\text{верхн.}}(x) - y_{\text{нижн.}}(x))dx = 8,425 \text{дел.}^2$

    \smallvspace

    \item Расчёт средней мощности $P$, расходуемой на перемагничивание образца: \\
    $P = \chi \cdot S_{\text{ПГ}} = 9,8 \cdot 10^{-4} \cdot 8,425 = 8,22 \cdot 10^{-3} \text{Вт}$

    \smallvspace

    \item Максимум магнитной проницаемости материала $\mu_{max} = 4655,66$ достигается при напряжённости поля $H = 48,657 \frac{\text{А}}{\text{м}}$
    
    
\end{itemize}
