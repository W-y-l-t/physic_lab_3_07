\text{Индукция магнитного поля в некотором материале}
\[
    \vec{B} = \mu_0(\vec{H} + \vec{J})
\]

\text{Где:}
\begin{center}
    $\vec{H} - \text{напряжённость магнитного поля}$ \\
    $\vec{J} - \text{намагниченность материала}$ \\
    $\mu_0 = 4\pi \cdot 10^{-7} \frac{\text{Гн}}{\text{м}} - \text{магнитная постоянная}$
\end{center}

\text{Зависимость магнитной проницаемости материала}
\[
    \mu = 1 + \frac{J}{H} = \frac{B}{\mu_0 H}
\]

\text{Теорема о циркуляции напряженности магнитного поля}
\[
    \oint H d\ell = N_1 \cdot I_1
\]

\text{Где:}
\begin{center}
    $N_1 - \text{число витков на намагничивающей (первичной) обмотке образца}$ \\
    $I_1 - \text{сила тока образца}$
\end{center}

\text{Напряженность магнитного поля}
\[
    H = \frac{N_1}{\ell} \cdot I_1 = \frac{N_1}{\ell R_1} \cdot K_x \cdot x = \alpha \cdot K_x \cdot x
\]

\text{Где:}
\begin{center}
    $x - \text{координата по горизонтальной оси OX экрана осциллографа}$ \\
    $K_x - \text{цена деления горизонтальной шкалы}$
\end{center}

\text{Электродвижущая сила во вторичной обмотке по Закону Фарадея}
\[
    \mathscr{E} = N_2 \left| \frac{\partial \Phi}{\partial t} \right| = N_2 S \left| \frac{dB}{dt} \right|
\]

\text{Где:}
\begin{center}
    $N_2 - \text{число витков на вторичной обмотке образца}$
\end{center}

\text{Индукция магнитного поля в образце}
\[
    B = \frac{1}{N_2 S} \int{\mathscr{E}}dt
\]

\text{Напряжение на интегрирующей RC-цепочке}
\[
    U_C = \frac{1}{R_2 C_1} \int{\mathscr{E}}dt
\]

\text{Связь напряжения конденсатора с магнитной индукцией}
\[
    U_C = \frac{N_2 S}{R_2 C_1} \cdot B
\]

\text{Связь индукции и вертикального размера осцилограммы}
\[
    B = \frac{R_2 C_1}{N_2 S}\cdot K_y \cdot y = \beta \cdot K_y \cdot y  
\]

\text{Где:}
\begin{center}
    $x - \text{координата по вертикальной оси OX экрана осциллографа}$ \\
    $K_x - \text{цена деления вертикальной шкалы}$
\end{center}

\text{Формула для расчета потерь энергии в ферромагнетике}
\[
    P = \chi \cdot S_{\text{ПГ}}
\]

\text{Где:}
\begin{center}
    $S_\text{ПГ} - \text{площадь петли гистерезиса}$
\end{center}

\text{Коэффициент $\chi$}
\[
    \chi = K_x K_y\frac{N_1 R_2 C_1}{N_2 R_1}f
\]

\text{Где:}
\begin{center}
    $f - \text{частота сигнала, подаваемого на первичную обмотку трансформатора}$
\end{center}

\text{Элементарная работа по перемагничиванию единицы объёма}
\[
    dA = V \cdot H dB
\]

\text{Полная работа по перемагничиванию единицы объема}
\[
    \frac{A}{V} = \frac{1}{V} \oint dA = \oint H dB
\]

\textbf{Исходные данные}
\begin{center}
    $N_1 = 1665$ - число витков на первичной обмотке образца. \\
    $N_2 = 970$ - число витков на вторичной обмотке образца. \\
    $\ell = 7,8 \pm 0,1$ см - средняя длина ферромагнетика. \\
    $S = 0,64 \pm 0,05$ см$^2$ - площадь поперечного сечения ферромагнетика. \\
    $R_1 = 68 \pm 10\%$ Ом \\
    $R_2 = 470 \pm 10\%$ кОм \\
    $C_1 = 0,47 \pm 10\%$ мкФ \\
    $f = 35$ Гц - частота сигнала, \\ подаваемого на первичную обмотку трансформатора
\end{center}